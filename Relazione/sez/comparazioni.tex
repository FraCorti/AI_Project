\section{Comparazioni}
Questa sezione sarà incentrata sul paragonare i risultati ottenuti delle 2 metodologie applicate, cercando anche di intuire il motivo e i fattori di squilibrio che hanno influito su di essi.
\subsection{Tra reti neurali}
I risultati in questo caso parlano chiaro, la rete neurale senza dropout ha prestazioni decisamente inferiori nelle prime iterazioni, anche se conclude con un risultato decisamente positivo. D'altro canto, come è stato sottolineato nelle sezioni precedenti, il dropout serve proprio ad evitare l'overfitting, cosa che, dato il dataset preso in considerazione, ha aiutato molto. Vediamo infatti come già alla quinta iterazione la rete neurale arrivi ad avere un errore che potremmo dire essere "fisiologico", questo conferma ancora una volta l'importanza di studiare il contesto e prendere gli opportuni accorgimenti.
\subsection{Reti neurali e Logistic regression}
Per quanto riguarda il confronto tra reti neurali e logistic regression, vi è una netta differenza nel caso dell'utilizzo o meno del dropout. Infatti, dai risultati emerge che la logistic regression è migliore quando non viene utilizzato il metodo correttivo, mentre peggiore altrimemti. Si nota soprattutto nella metrica di accuracy, nella quale troviamo la logistic regression leggermente al di sotto delle reti neurali con dropout, ma di gran lunga superiore alla rete neurale che non utilizza il dropout.
\subsection{Reti neurali perchè migliori?}
Abbiamo capito quindi che la discriminante in questo caso è l'utilizzo di azioni preventive e correttive sul dataset e sul rischio di overfitting, detto questo sottolineamo che il dataset è stato appositamente scelto per far risaltare questa differenza nel confronto finale.

